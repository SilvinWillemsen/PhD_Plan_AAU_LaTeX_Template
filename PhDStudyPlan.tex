\section{Project summary / abstract}

A short (max 400 words) summary in layman’s terms describing key motivation, significance, methodology, and expected outcome of the PhD study. A reader of the local newspaper should be able to understand the summary. 
\\
\\
\textit{12-month plan: An updated version of the summary.}  

\section{The scientific content of the PhD project}

\subsection{Background}
The background for the project problem should be described (corresponding to maximum 300 words).

\subsection{State-of-the-art}
An introduction stating the state-of-the-art for the PhD project. The introduction should include key references listed under section 11. Typically, at least 10-15 references to peer reviewed scientific material are expected. In case it is necessary to refer to non-peer reviewed material then use a footnote (or parenthesis) to provide information to the source. 
\\
\\
\textit{12-month plan: The state-of-the-art for the PhD project must be updated including use of the most essential references (list references under section 11).}

\subsection{Project objectives}
Statement of the project’s objectives followed by a formulation of the specific problem(s) addressed in the study. This could be formulated as a hypothesis and or research questions if applicable. Explain the relevance of the present PhD project so the scientific contribution will be evident – i.e. explain how the project advances current state-of-the-art. Scientific challenges should be clearly defined – don’t mistake this for technological challenges.
\\
\\
\textit{12-month plan: Update project objectives.}

\subsection{Key methods}
Coverage of the methodological needs, identification of means of meeting these needs, and the methodological design. The coverage should include techniques for evaluating or assessing the outcomes of the project. Some examples of methodology are empirical studies (observational or experimental), statistical analyses, mathematical deduction, computer simulation...
\\
\\
\textit{12-month plan: Update the key methods for the PhD project.}

\subsection{Significance and outcome}
Potential significance and application(s) of the project’s expected outcome, possibly including methodological contributions. 
\\
\\
\textit{12-month plan: Experiences and results obtained so far in the project followed by expected outcome of the entire PhD project.} 

\section{Work and publication plans}\label{sec:plans}
\subsection{Work and Time Plans}
Work and time plans including measurable milestones (project milestones and deadlines for expected publications for each quarter or finer). It is recommended that a number of sub-project activities are identified that can be associated with milestones, so that there are milestones (at least) each six months during the project. Remember to allocate time for preparing scientific publications (conference papers, journal papers etc.). Deadlines for the expected publications must be included. These milestones will allow the PhD student and supervisor(s) to assess the status of the project each six months and to revise the plan if needed. The specific activities described in the time plan must be of such detail that it is clear what should be carried out. A proposal for the layout of a time schedule is shown below. Assess the risk of not reaching the various milestones at the deadlines given. Provide precautions for milestones for which the completion of the associated task could be problematic.
\\
\\
\scriptsize
\def \tabcolwidth {0.030\columnwidth}
\begin{tabular}{|c|p{\tabcolwidth}|p{\tabcolwidth}|p{\tabcolwidth}|p{\tabcolwidth}|p{\tabcolwidth}|p{\tabcolwidth}|p{\tabcolwidth}|p{\tabcolwidth}|p{\tabcolwidth}|p{\tabcolwidth}|p{\tabcolwidth}|p{\tabcolwidth}|}
\hline
    \bf Year & \multicolumn{2}{c|}{\bf 2018} & \multicolumn{4}{c|}{\bf 2019} &\multicolumn{4}{c|}{\bf 2020}& \multicolumn{2}{c|}{\bf 2021} \\
    \hline
    \centering\bf Quarter & \centering\bf 3 & \centering\bf 4 & \centering\bf 1 & \centering\bf 2 & \centering\bf 3 & \centering\bf 4 & \centering\bf 1 & \centering\bf 2 & \centering\bf 3 & \centering\bf 4 & \centering\bf 1 & \multicolumn{1}{c|}{\bf 2}  \\
    \hline
    Literature study & \cellcolor{green} & \cellcolor{green} & \cellcolor{green} & \cellcolor{yellow} & & & & & & & &  \cellcolor{lighterblue}\\
    \hline
    Design of setups for electrical & & \cellcolor{green} & \cellcolor{green} & \cellcolor{yellow} & & & & & & & & \cellcolor{lighterblue}\\
    and combined tests of materials &  & \cellcolor{green} & \cellcolor{green} & \cellcolor{yellow} & & & & & & & &  \cellcolor{lighterblue}\\
    \hline
    Test on composite materials’ electrical & & \cellcolor{green} & \cellcolor{green} & \cellcolor{yellow} & \cellcolor{red}& & & & & & & \cellcolor{lighterblue}\\
    properties and combined tests &  & \cellcolor{green} & \cellcolor{green} & \cellcolor{yellow} &\cellcolor{red} & & & & & & &  \cellcolor{lighterblue}\\
    \hline
    Design of setups for small-scale & & & \cellcolor{green} & \cellcolor{yellow} & & & & & & & & \cellcolor{lighterblue}\\
    lightning shielding simulation test &  &  & \cellcolor{green} & \cellcolor{yellow} & & & & & & & &  \cellcolor{lighterblue}\\
    \hline
    Small-scale lightning shielding & & & & & \cellcolor{red} & \cellcolor{red} & & & & & & \cellcolor{lighterblue}\\
    simulation test &  & & & & \cellcolor{red} & \cellcolor{red} & & & & & &  \cellcolor{lighterblue}\\
    \hline
    Development of system critical parts for& & & & & \cellcolor{red} & \cellcolor{red} & & & & & & \cellcolor{lighterblue}\\
     small/full-scale HV test on cross-arm &  & & & & \cellcolor{red} & \cellcolor{red} & & & & & &  \cellcolor{lighterblue}\\
    \hline
    Electrical-and-mechanical combined & & & & & & \cellcolor{red} & \cellcolor{red} & & & & & \cellcolor{lighterblue}\\
    test on small parts of composite cross-arm & & & & & & \cellcolor{red} & \cellcolor{red} & & & & &  \cellcolor{lighterblue}\\
    \hline
    Test of cross-arm phase-to-phase/ground & & & & & & & \cellcolor{red} & \cellcolor{red} & & & &  \cellcolor{lighterblue}\\
    insulation (LI, SI and AC in wet/dry) & & & & & & & \cellcolor{red} & \cellcolor{red} & & & &  \cellcolor{lighterblue}\\
    \hline 
    Test and measurement of corona & & & & & & & & & \cellcolor{red} & \cellcolor{red} & &  \cellcolor{lighterblue}\\
    activities on cross-arm (dry/moisture) & & & & & & & & & \cellcolor{red} & \cellcolor{red} & &  \cellcolor{lighterblue}\\
    \hline 
    Test on lightning protection & & & & & & & & & \cellcolor{red} & \cellcolor{red} & &  \cellcolor{lighterblue}\\
    performance of the pylon & & & & & & & & & \cellcolor{red} & \cellcolor{red} & &  \cellcolor{lighterblue}\\
    \hline 
    Writing thesis & & & & \cellcolor{red}&\cellcolor{red} & & &\cellcolor{red} &\cellcolor{red} &\cellcolor{red} & \cellcolor{red} &\cellcolor{lighterblue}
    \\
    \hline PhD Courses & & \cellcolor{green} & & & \cellcolor{red} & & \cellcolor{red} & \cellcolor{red} & & \cellcolor{red} & & \cellcolor{lighterblue}
    \\
    \hline 
    Publishing papers & & & \centering C1 & & \centering C2 & \centering C3 &\centering J1&\centering C4 & \centering C5 & \centering J2 & \centering J3 &  \cellcolor{lighterblue}
    \\
    \hline
    Milestones& & & & & {\scriptsize MS1} & \centering {\scriptsize MS2} &  \centering {\scriptsize MS3} & \centering {\scriptsize MS4} & & \centering {\scriptsize MS5} & \centering {\scriptsize MS6} & \cellcolor{lighterblue}
    \\
    \hline
\end{tabular}
\vspace{0.3cm}
\\
\begin{tabular}[h]{|c|c|c|c|}
\hline
\cellcolor{green} Activities finished & \cellcolor{yellow} Activities being performed & \cellcolor{red} Planned activities & \cellcolor{lighterblue} Buffer time\\
\hline
\end{tabular}
\vspace{0.3cm}
\\
\normalsize
Milestones:
\\
{\small MS1: Material selection. MS2: Combined test.  MS3: Small-scaled lightning protection test. MS4: Tests to verify insulation capacity. MS5: Tests to verify dimensions. MS6: handing the thesis}
\\
\\
\textit{12-month plan: An updated time schedule for the entire project must be included.}

\subsection{Outline of thesis}
Outline the content of the thesis, including an indication on whether the thesis is expected to take the form of a collection of papers (which is generally recommended by the Doctoral School) or a monograph. This description could be organized by means of an overall table of contents. In case of a collection of papers, the thesis must contain an extended summary (e.g. 20-40 pages) that provides an overview of the topic, reviews the papers, highlights the most significant scientific results achieved, and relates the findings to the current international state-of-the-art. Note that for each paper on which the thesis is based, a co-author statement must finally be submitted together with the thesis.
\\
\\ 
\textit{12-month plan: update outline of content of thesis.}

\subsection{Tentative publication list}
Provide tentative list of publications. Regardless of the form of the thesis, it is recommended that results are documented and submitted for publication in peer reviewed outlets throughout the project. For each publication, the following should be indicated or estimated: working title, co-authors, length in pages, outlet (e.g. a named conference or journal), and approximate time of submission. Indicate who has the primary responsibility for the publication. Publications in journals indexed in the Danish “Autoritetslister” and Web of Science are encouraged (use the ISI service at http://apps.isiknowledge.com/).
\\
\\
\textit{12-month plan: update list of papers.}

\section{Supervisor/student co-operation agreements}
Agreement on the relationship between supervisor and student (meeting frequency, communication forms, mutual expectations, etc.). The Doctoral School expects that the supervisor and student should meet (face to face if possible) at least once every month. See note on the homepage of the Doctoral School for inspiration: http://www.phd.tech.aau.dk/current-students/forms-templates/. 
\\
\\
It is necessary that the supervisor and the student conduct a meeting in which the mutual expectations are clarified before authoring this section.
\\
\\
\textit{12-month plan: Status for relationship and updated agreement on the relationship between supervisor and student.}

\section{Plan for PhD Courses (both general and project related courses)}
Courses adding up to 30 ECTS credits must be outlined. The estimated workload for the student is 28 hours per ECTS credit. Please use the table below: 

\begin{tabular}{|p{5cm}|l|l|l|l|}
\hline
   \centering Courses  & Place/organised by & ECTS & General / & Status \\
   & & & Project Course & \\
   \hline
   
\rowcolor{lightestblue} \bf  & & & & \\ 
& & & & \\
\rowcolor{lightestblue} \bf  & & & & \\ 
& & & & \\
\rowcolor{lightestblue} \bf  & & & & \\ 
& & & & \\
\rowcolor{lightestblue} \bf  & & & & \\ 
& & & &\\
    \hline
\end{tabular}
\\
\\
General courses are those offered by the Faculty’s Doctoral School or other doctoral schools as courses of general interest for the PhD process, and project related courses are those offered by the Faculty’s Doctoral Programmes or other relevant doctoral programmes, directed to the particularities of the student’s project. Each type of courses should generally cover a minimum of 10 ECTS. In case of deviation, specific reasons for this should be given. Indicate topics of relevance to be covered if specific course names are not yet available. All courses must be at PhD level at identifiable institutions. In general, no single course should exceed 6 ECTS credit points.
\\
\\
If conference and workshop participation is part of the planned courses, each such participation must be accompanied by a written report by the PhD student that relates the specific activity to the PhD project. This report must be of general value for the project. A template can be found at the homepage of the Doctoral School (http://www.phd.tech.aau.dk/current-students/forms-templates/). The report is to be submitted to and approved by the supervisor.
\\
\\
A study group can also count as part of the general/project (whatever applies) course load. However, the study group activity has to be approved by the head of the programme that the student is affiliated with. This approval is based on a description of the study group activity written by the student and the supervisor. The student is required to keep a copy of the description. Study groups should in general not cover more than 6 ECTS. However, the head of the doctoral school can allow further ECTS for study groups if deemed appropriate.
\\
\\
Course activities that relate to other activities, other than courses, i.e., workshop, tutorials, conference participation, study groups must not exceed 6 ECTS credits.
\\
\\
Only courses at PhD level are approved. If a course at master level is deemed to be highly relevant for the PhD project, the supervisor can establish a study group on the topic, which includes the master course and additional reading/discussion to bring it up to PhD level. The written report mentioned above on participation in a study group must be completed to get course credit. To ensure the scientific level, the study circle must be headed by a member of the scientific staff, who is Professor or Associate Professor (senior scientist level). 
\\
\\
By completion of the PhD study, documentation of the contents and the extent of the courses must be provided along with approval from the main supervisor. 
\\
\\
\textit{12-month plan: Update the course table.}

\section{Plan for fulfilment of knowledge dissemination}
Plan for dissemination of knowledge and findings from the project (e.g., in newspaper articles, seminars, conference presentations, teaching etc.). As seen, dissemination is not only teaching but can also be other activities. Moreover, it should be described how the knowledge will be disseminated to relevant organizations and industry and to the general public.
\\
\\
\textit{12-month plan: update plan for dissemination.}


\section{Agreements on immaterial rights to patents}
Outline relevant agreements on immaterial rights to patents, etc. produced during the PhD project. Typically, it is sufficient to mention that IPR is handled via the standard university rules.
\\
\\  
\textit{12-month plan: Update this section if applicable.}

\section{External co-operation}
Describe the plan for fulfilment of the legal requirements concerning the contents of the PhD studies, as specified in the law, pertaining to “engagement in active research environments, including stays at others primarily foreign research institutions, private research organizations, etc.” It is recommended that this be achieved via one or more stays at a foreign research institution with a total duration of 3-6 months. One or two tentative co-operative institutions must be described. The co-operation must be an active research co-operation in which also the host institution contributes to the research. The host institution must be a research institution or a company doing research. Summer schools, conference attendance etc. are not considered external cooperation. 
\\
\\
\textit{
12-month plan: The description must be updated with completed and expected/planned co-operation activities. At this point, these should be specific and the host should have identified. Note that it is very important that the external stay is planned well in advance in order for the PhD student to benefit the most.} 

\section{Career Plan (12-month plan only)}
\textit{12-month plan: Describe your long-term career plans, i.e., beyond the PhD studies. For example, do you plan to pursue a career in academia and, if so, what is the next step after graduation? Is it a postdoc abroad or an industrial postdoc after which you plan to become assistant professor? Or do you intend to become an industrial researcher, and, if so, in what industry and with what potential companies. In what role do you see yourself long-term. Do you, for example, see yourself as a technical specialist or is your ambition to become a research manager? Explain how your PhD study plan and the choices you have made herein supports your career plan (e.g., the courses you plan to follow, your plans for external collaborations and knowledge dissemination).}


\section{Financing budget}
Information on the financing budget for the PhD project i.e. expenses needed to complete the project (not salary). The funding source or sources should be identified. This part is for information entirely and cannot be used to demand any resources from the department – this part is governed by the specific agreement between the department and the PhD student, which is agreed upon by the time of enrolment.
\\
\\
\textit{12-month plan: Update this section if applicable.}

